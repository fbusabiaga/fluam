\documentclass[aps,prl]{revtex4}
\usepackage{graphicx}
\usepackage{amsmath}

\newcommand{\bd}{\boldsymbol}

\begin{document}
\setlength{\parindent}{0cm}

\title{ Gblob Manual}


\section{Fluctuating hydrodynamics on GPUs}
\subsection{Compile}
Edit the two firsts lines in the makefile, and type make. Note, this code needs the
``HydroGrid'' code from Donev to calculate the structure factors.
\subsection{How to run}
Type the executable name following with the input file \\ \\
\emph{mainstandard inputfile} \\ \\
If a input file is not provided the code will search the file data.main.

\subsection{Input File}
The input file (data.main by default) contains the parameters for the 
simulation. The program doesn't use internal units, if the parameters are coherent the results
should be fine. The input file should/might contain the following:

$\bullet$ Comments: start with the character \#. \\

$\bullet$ Scheme: the code supports several schemes, chose only one between \\
RK3: RK3, without thermostat or concentration. Use \\
\emph{}\#nothing because RK3 is the default scheme \\ \\
thermostat: RK3 with thermostat, without concentration. Use \\
\emph{thermostat} \\ \\
ghost: RK3 without thermostat or concentration. It's 
implemented using ghost cells and it's slower than the
scheme RK3. Use \\
\emph{ghost} \\ \\
binaryMixture: RK3 with thermostat and concentration. It's 
implemented using ghost cells. Use \\
\emph{binaryMixture} \\ \\
binaryMixtureWall: like binaryMixture but with a hard wall in 
the planes y=0 and y=ly. Use \\
\emph{binaryMixtureWall} \\ \\
giantFluctuations: it's like binaryMixtureWall but with a different boundary conditions and
the term $\nabla (\rho D cS_T \nabla T)$ in the concentration equation. \\
\emph{giantFluctuations} \\ \\
continuousGradient: it's like binaryMixture but with the additional term 
$\frac{1}{2}(v_{\bd{i}+\bd{y}/2}^y+v_{\bd{i}-\bd{y}/2}^y)\rho_{\bd{i}}\overline{\nabla c}$
in the equation for the density of the species 1. Provide
the constant gradient in the input file with the parameter \emph{gradTemperature}. \\
\emph{continuousGradient} \\


$\bullet$ Fluid density \\
\emph{densfluid               0.632} \\ \\

$\bullet$ shearviscosity \\
\emph{shearviscosity          53.71} \\ \\
$\bullet$ bulkviscosity. Note that the term in grad(div(v)) is (bulkviscosity+shearviscosity/3) \\
\emph{bulkviscosity           127.05} \\ \\
$\bullet$ Pressure. In the code the pressure is a second order polynomial of the density
$p = a0 + a1*\rho + a2*\rho^2$. Give the parameters a0, a1 and a2 \\
\emph{\#pressureparameters a0 a1 a2} \\ 
\emph{pressureparameters 231.091   -947.837   920.671} \\ \\
$\bullet$ Temperature. The temperature is in fact ($k_BT$) with the appropiate units of 
energy \\
\emph{temperature} 249.4 \\ \\
$\bullet$ Number of time steps \\
\emph{numsteps                1000} \\ \\
$\bullet$ Number of time step in which the code won't save data. \\
\emph{numstepsRelaxation	0} \\ \\
$\bullet$ Time step \\
\emph{dt                     	3.408} \\\\
$\bullet$ Sample frequency \\
\emph{samplefreq              1} \\\\
$\bullet$ Initialize fluid \\
\emph{initfluid option} \\
with \\
option=0: fluid at rest, vx=vy=vz=0 and $\rho=cte$ \\
option=1: close to thermal equilibrium, $<v_i(\mathbf{r})>=0$, 
$<v_i(\mathbf{r}) v_j(\mathbf{s})>=(k_BT/\rho V_{cell}) \delta_{ij} \delta_{rs}$ and $\rho=cte$ \\
In the simulation with concentration this initializes like $<concentration(y)>_{thermal equilibrium}$ \\ \\

$\bullet$ Output name. The output files created by the code start with ``outputname''. It's
possible indicate the directory \\
\emph{outputname		../data/run1} \\\\

$\bullet$ Number of cells mx my mz \\
\emph{cells			32	32	32} \\\\

$\bullet$ Size system lx ly lz \\
\emph{celldimension		8000	8000	8000} \\\\


$\bullet$ Random number seed. Default seed=time(0), that means that the code takes a seed from the
computer's clock, but you can provide your own seed \\
\emph{seed			4} \\ \\

$\bullet$ Set device, In general it's better don't use this option. \\
\emph{setDevice                 numberDevice} \\
\emph{setDevice                 1} \\

$\bullet$ Background velocity \\
\emph{backgroundvelocity vx0 vy0 vz0} \\

The following options are only relevant for the binaryMixture or binaryMixtureWall schemes. \\

$\bullet$ Diffusion. For a ideal gas the diffusion is diffusion=A/$\rho$. Here you have to
provide A \\
\emph{diffusion	     3.66} \\ \\

$\bullet$ Mass species 0 \\
\emph{massSpecies0	18}\\\\
$\bullet$ Mass species 1 \\
\emph{massSpecies1	18}\\\\
$\bullet$ Concentration for the species 1 \\
\emph{concentration 	0.5}\\\\

The following options are only relevant for the binaryMixtureWall scheme. \\
$\bullet$ Concentration in the walls y=0 and y=ly \\
\emph{concentrationWall           0.55     0.45} \\ \\
$\bullet$ Velocities in the walls y=0 and y=ly \\
\emph{vxWall			0 	0}\\\\
\emph{vzWall			0	0}\\\\

The following options are only relevant for the giantFluctuations scheme. \\
$\bullet$ Soret coefficient \\
\emph{soretCoefficient        0.06486} \\ \\
$\bullet$ Temperature gradient \\
\emph{gradTemperature         174}\\\\


\subsection{Tips to modify the code}
$\bullet$ Use the cuda compiler ``nvcc''. \\

$\bullet$ Take a look at the file main.cpp and follow the flow for one scheme, in the
most part they are similar. \\

$\bullet$ The variables (i.e. vx, vy...) are duplicate, one copy resides in the CPU and the
other in the GPU. They are not synchronize, you have to copy explicitly the variable
from the CPU to the GPU or viceversa. The copy process is slow try to avoid copies. \\

$\bullet$ If different cuda functions (``kernels'') use the same variable they and the variable 
must be declare in the same file. Take a look at the file GPU.cu.




\end{document}
